\documentclass[12pt]{ctexart}
\usepackage[margin=3cm]{geometry}
\linespread{1.5}
\title{《薛兆丰经济学讲义》读书笔记}
\date{}
\begin{document}
\maketitle
\section{稀缺\quad 为何商业是最大的慈善}
\subsection{战俘营里的经济组织}
有人的地方就有交易。

有交易就有价格,有价格就会有价格波动。

有了交易就会有对货币的需求,有了货币就会有通货膨胀和通货紧缩。
\subsection{马粪争夺案}
鼓励创造财富好于鼓励对财富做标记,后者使得人们耗费过多精力用于看管财富,丧失了创造和积累财富的积极性。

公正的背后是效率的考量,不是个人的效率,而是整体社会发展的长远效率。
\subsection{看得见的和看不见的}
好的经济学家能够同时权衡看得见和看不见的后果。

看见看不见的东西需要想象力。

决策的时候需要看到暂时/永远看不到的东西。
\subsection{区分愿望和结果}
现代经济学和其他研究财富的学问的区别:其他人研究“事与愿符”的规律,而经济学研究的是“事与愿违”的规律。

那些一眼就能看出好坏的思想,我们能够识别,能够抵制。倒是那些用良好愿望包装起来的思想,我们比较难识别。

政府立法不是解决问题的终点:因为人的主观能动性,每个人会产生不同的对策,结果会和预期有很大出入。这背后是某些不以人的意志为转移的客观规律。
\section{}
\end{document}